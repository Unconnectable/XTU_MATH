\documentclass{article}
\usepackage{amsmath, amssymb}
\usepackage{geometry}
\usepackage{ctex} % 支持中文
\usepackage{tcolorbox} % 用于添加颜色框
\usepackage{hyperref} % 用于超链接
\geometry{a4paper, margin=1in}
\title{2023 级数学韶峰班选拔考试 \\ 数学分析试卷}
\author{\href{https://github.com/Unconnectable}{filament}} % 加入作者超链接
\date{2024 年 8 月 28 日}

\begin{document}

\maketitle

\section*{考试方式:闭卷}
\section*{考试时间:2 小时}

\begin{enumerate}
    \item (20 分) 
    \begin{tcolorbox}[colframe=blue!50!black, colback=blue!5!white]
    研究级数 $\sum_{n=1}^{\infty} \frac{\cos(n x)}{n}$ 的敛散性,其中 $x \in \mathbb{R}$.
    \end{tcolorbox} \\\\\\\\
    
    \item (20 分) 
    \begin{tcolorbox}[colframe=green!50!black, colback=green!5!white]
    设 $I$ 为有限长度的区间。证明:区间 $I$ 上实函数 $f(x)$ 一致连续的充要条件是 $f(x)$ 把 Cauchy 列映成 Cauchy 列,即若 $\left\{x_{n}\right\}$ 是 $I$ 上 Cauchy 列,则 $\left\{f\left(x_{n}\right)\}$ 也是 Cauchy 列。
    \end{tcolorbox} \\\\\\\\
    
    \item (20 分) 
    \begin{tcolorbox}[colframe=red!50!black, colback=red!5!white]
    讨论反常积分 $\int_{0}^{+\infty} \frac{\sin \left(x+\frac{1}{x}\right)}{x^{p}} \mathrm{d}x$ 的敛散性(包括绝对收敛、条件收敛、发散)。
    \end{tcolorbox} \\\\\\\\
    
    \item (20 分) 
    \begin{tcolorbox}[colframe=purple!50!black, colback=purple!5!white]
    设 $u_{n}(x), v_{n}(x)$ 在 $(a, b)$ 上连续,且 $\left|u_{n}(x)\right| \leq v_{n}(x), n=1,2, \cdots$。证明:若 $\sum_{n=1}^{\infty} v_{n}(x)$ 在 $(a, b)$ 上收敛于一个连续函数,则 $\sum_{n=1}^{\infty} u_{n}(x)$ 在 $(a, b)$ 上也收敛于一个连续函数。
    \end{tcolorbox} \\\\\\\\
    
    \item (20 分) 
    \begin{tcolorbox}[colframe=orange!50!black, colback=orange!5!white]
    证明 Stirling 公式: 
    \[
    n! \sim \sqrt{2 \pi n}\left(\frac{n}{e}\right)^{n} \quad (n \to \infty).
    \]
    \end{tcolorbox}
\end{enumerate}

\end{document}
